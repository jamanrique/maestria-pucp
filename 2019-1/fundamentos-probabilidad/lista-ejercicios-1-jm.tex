\documentclass[11pt]{article}
\usepackage[spanish]{babel}
\usepackage[utf8]{inputenc}

\title{Lista de ejercicios 1 - Fundamentos de Probabilidad}

\begin{document}
\maketitle
\author{Justo Andrés Manrique Urbina - 20091107}

\section{Pregunta 5.a.}

Demuestre que $\mathit{A}_{1} \in \sigma(C), \mathit{A}_{2} \in \sigma(C), ...$

\textbf{Demostración:}

Supongamos $\mathit{A}_{i} \in C, i=1,2, ...$

Recordemos que, por definición 1.2. del texto de la clase, $C \subset \sigma(C)$

Por lo tanto, $\mathit{A}_{i} \in C \subset \sigma(C), i=1,2, ...$

Finalmente, $\mathit{A}_{i} \in \sigma(C), i=1,2, ...$

\section{Pregunta 5.b.}
Demuestre que $\bigcup^{\infty}_{j=1} A_{j} \in \sigma(C)$.

\textbf{Demostración:}

Supongamos que $\mathit{A}_{i} \in \sigma(C), i=1,2, ...$,

Por definición de $\sigma$-álgebra (definición 1.1 del texto de la clase), toda $\sigma$-álgebra es cerrada respecto a reuniones infinitas enumerables.

Dada la suposición y la definición de $\sigma$-álgebra, se concluye que  $\bigcup^{\infty}_{j=1} A_{j} \in \sigma(C)$.

\section{Pregunta 16.a.}
Demuestre que $\mathit{f}^{-1} (\sigma(C))$ es una $\sigma$-álgebra en $\Omega_{1}$.

\textbf{Demostración:}

Si $\sigma(C)$ es una $\sigma$-álgebra en $\Omega_{2}$, entonces se cumplen las siguientes definiciones:

\begin{itemize}
	\item $\Omega_{2} \in \sigma(C)$
	\item $\forall \mathit{A} \in \sigma(C): \mathit{A}^{c} \in \sigma(C)$
	\item $\forall \mathit{A}_{1},\mathit{A}_{2}, ... \in \sigma(C): \bigcup^{\infty}_{j=1} A_{j} \in \sigma(C)$
\end{itemize}

En base a ello, aplicamos la imagen inversa a la $\sigma$-álgebra generada y verificaremos si se cumple la definición en $\Omega_{1}$.

\begin{itemize}
	\item $\mathit{f}^{-1}(\Omega_{2}) = \Omega_{1} \in \mathit{f}^{-1}(\sigma(C))$
	\item $\forall \mathit{f}^{-1}(\mathit{A}) \in \mathit{f}^{-1}(\sigma(C)): \mathit{f}^{-1}(\mathit{A}^{c}) = (\mathit{f}^{-1}(\mathit{A}))^{c} \in \mathit{f}^{-1}(\sigma(C))$
	\item $\forall \mathit{f}^{-1}(\mathit{A}_{1}),\mathit{f}^{-1}(\mathit{A}_{2}), ... \in \mathit{f}^{-1}(\sigma(C)): \mathit{f}^{-1}(\bigcup^{\infty}_{j=1} A_{j}) = \bigcup^{\infty}_{j=1}(\mathit{f}^{-1}(A_{j})) \in \mathit{f}^{-1}(\sigma(C))$
\end{itemize}

Observamos que, aplicando la imagen inversa a $\sigma(C)$, la definición de $\sigma$-álgebra se mantiene en $\Omega_{1}$. 

Por lo tanto, $\mathit{f}^{-1} (\sigma(C))$ es una $\sigma$-álgebra en $\Omega_{1}$.

\section{Pregunta 16.b.}

Demuestre que  $\mathit{f}^{-1} (C) \subset \mathit{f}^{-1} (\sigma(C))$

\textbf{Demostración:}

Recordemos que:
\begin{itemize}
\item $\mathit{f}^{-1} (C) = \{\mathit{f}^{-1} (A): A \in C\}$
\item $\mathit{f}^{-1} (A) = \{w \in \Omega_{1}: \mathit{f}(w) \in A\}$
\end{itemize}

Debido a ello, se obtiene que $\mathit{f}(w) \in A \in C$

Por definición, $C \subset \sigma(C)$

Por lo tanto, $\mathit{f}(w) \in A \in C \subset \sigma(C)$

Aplicando imagen inversa, se obtiene que: $w \in \mathit{f}^{-1} (A) \in \mathit{f}^{-1} (C) \subset \mathit{f}^{-1} (\sigma(C))$

Finalmente, se observa que $\mathit{f}^{-1} (C) \subset \mathit{f}^{-1} (\sigma(C))$.



\end{document}