\documentclass[11pt]{article}
\usepackage[spanish]{babel}
\usepackage[utf8]{inputenc}

\title{Lista de ejercicios 1 - Fundamentos de Probabilidad}

\begin{document}
\maketitle
\author{Justo Andrés Manrique Urbina - 20091107}

\section{Pregunta 5.a.}

Demuestre que $\mathit{A}_{1} \in \sigma(C), \mathit{A}_{2} \in \sigma(C), ...$

\textbf{Demostración:}

Supongamos $\mathit{A}_{i} \in C, i=1,2, ...$

Recordemos que, por definición 1.2. del texto de la clase, $C \subset \sigma(C)$

Por lo tanto, $\mathit{A}_{i} \in C \subset \sigma(C), i=1,2, ...$

Finalmente, $\mathit{A}_{i} \in \sigma(C), i=1,2, ...$

\section{Pregunta 5.b.}
Demuestre que $\bigcup^{\infty}_{j=1} A_{j} \in \sigma(C)$.

\textbf{Demostración:}

Supongamos que $\mathit{A}_{i} \in \sigma(C), i=1,2, ...$,

Por definición de $\sigma$-álgebra (definición 1.1 del texto de la clase), toda $\sigma$-álgebra es cerrada respecto a reuniones infinitas enumerables.

Dada la suposición y la definición de $\sigma$-álgebra, se concluye que  $\bigcup^{\infty}_{j=1} A_{j} \in \sigma(C)$.

\section{Pregunta 16.a.}
Demuestre que $\mathit{f}^{-1} (\sigma(C))$ es una $\sigma$-álgebra en $\Omega_{1}$.

\textbf{Demostración:}

Supongamos que $\sigma(C)$ es una $\sigma$-álgebra en $\Omega_{2}$. Entonces, se cumplen las siguientes propiedades:

\begin{itemize}
	\item $\Omega_{2} \in \sigma(C)$
	\item $\forall \mathit{A} \in \sigma(C): \mathit{A}^{c} \in \sigma(C)$
	\item $\forall \mathit{A}_{1},\mathit{A}_{2}, ... \in \sigma(C): \bigcup^{\infty}_{j=1} A_{j} \in \sigma(C)$
\end{itemize}

Para demostrar que $\mathit{f}^{-1}(\sigma(C))$ es una $\sigma$-álgebra en $\Omega_{1}$, verificaremos si dicha imagen inversa cumple los 3 axiomas de la definición de $\sigma$-álgebra en $\Omega_{1}$. 

Respecto al primer axioma, recordemos que $\Omega_{1} = \mathit{f}^{-1}(\Omega_{2})$ y que $\Omega_{2} \in \sigma(C)$. Por lo tanto, por definición, $\Omega_{1} \in \mathit{f}^{-1}(\sigma(C))$. Se cumple el primer axioma.

Respecto al segundo axioma, recordemos que $(\mathit{A})^{c} \in \sigma(C)$, y si $\mathit{f}^{-1}(A) \in \mathit{f}^{-1}(\sigma(C))$, entonces, por propiedad, $\mathit{f}^{-1}(A^{c}) = (\mathit{f}^{-1}(A))^{c} \in \mathit{f}^{-1}(\sigma(C))$. Se cumple el segundo axioma.

Respecto al tercer axioma, recordemos que $\bigcup^{\infty}_{j=1} A_{j} \in \sigma(C)$ y que si $\mathit{f}^{-1}(A_{j}) \in \mathit{f}^{-1}(A), j=1,2, ... $, entonces $\bigcup_{j=1}^{\infty} \mathit{f}^{-1}(A_j) = \mathit{f}^{-1}(\bigcup_{j=1}^{\infty}A_j) \in \mathit{f}^{-1}(\sigma(C))$. Se cumple el tercer axioma.

Dado que se cumplen los tres axiomas en $\Omega_{1}$ se concluye que $\mathit{f}^{-1} (\sigma(C))$ es una $\sigma$-álgebra en $\Omega_{1}$.

\section{Pregunta 16.b.}

Demuestre que $\mathit{f}^{-1} (C) \subset \mathit{f}^{-1} (\sigma(C))$.

\textbf{Demostración:}

Supongamos $\sigma(C)$ es una $\sigma$-álgebra de $\Omega_2$ generada en $C$. Por lo demostrado en la pregunta anterior, $\mathit{f}^{-1} (\sigma(C))$ es una $\sigma$-álgebra en $\Omega_{1}$.

Por definición, la $\sigma$-álgebra generada en $C$ es la intersección de todas las $\sigma$-álgebras que contienen a $C$. 

Por lo tanto, dado que $\mathit{f}^{-1} (\sigma(C))$ es una $\sigma$-álgebra en $\Omega_{1}$ generada en $\mathit{f}^{-1}(C)$, entonces dicha $\sigma$-álgebra es la intersección de todas las $\sigma$-álgebras de $\Omega_{1}$ que contienen a $\mathit{f}^{-1}(C)$.

Dado lo indicado anteriormente, se concluye que $\mathit{f}^{-1} (C) \subset \mathit{f}^{-1} (\sigma(C))$.

\section{Pregunta 16.c.}

Demuestre que $\sigma(\mathit{f}^{-1}(C)) \subset \mathit{f}^{-1}(\sigma(C))$.

\textbf{Demostración:}

Supongamos $\sigma(C)$ es una $\sigma$-álgebra de $\Omega_2$ generada en $C$. Por lo demostrado en las preguntas anteriores, $\mathit{f}^{-1} (\sigma(C))$ es una $\sigma$-álgebra en $\Omega_{1}$ y $\mathit{f}^{-1} (C) \subset \mathit{f}^{-1} (\sigma(C))$.





\end{document}