\documentclass{article}
\usepackage[spanish]{babel}
\usepackage[utf8]{inputenc}
\usepackage{amsmath}
\usepackage{amsthm}
\usepackage{amsfonts}
\newtheorem{mydef}{Definition}
\newtheorem{mythm}{Theorem}
\newtheorem{myprf}{Proof}
\title{Práctica Calificada 1: Modelos Lineales 2}
\author{Justo Andrés Manrique Urbina}
\begin{document}
\maketitle
\textbf{Nota: } Utilizar el código R para observar los gráficos.
\section{Pregunta 2.a}

En el diagrama de dispersión se puede observar lo siguiente:
\begin{itemize}
	\item Se observa que existe una relación negativa entre el salario y la inestabilidad emocional, al fragmentar el conjunto de observaciones por nivel de educación.
	\item Se observa que tener una educación media tiene relación con un menor salario en promedio; y que tener educación baja o alta tiene relación con un mayor salario, siendo (en promedio) la educación baja la que tiene mayor salario.
	\item Se observa que tener una educación media tiene relación con menor inestabilidad emocional en promedio; y que tener educación baja o alta tiene relación con una mayor inestabilidad emocional, siendo las personas con educación baja la que (en promedio) tienen mayor inestabilidad emocional.
\end{itemize}

\section{Pregunta 2.b}

En el análisis de regresión a través de la función \textit{summary} en R, se observó lo siguiente:
\begin{itemize}
	\item Se observó que un aumento en una unidad del índice de inestabilidad emocional tiene relación con una disminución del salario (el estimado puntual es de -3177).
	\item Se observó que una disminución en el nivel de educación (de alto a medio o bajo) tiene relación con una disminución del salario, siendo el mayor impacto de alto a bajo.
\end{itemize}

Los resultados contradicen lo observado en la segunda viñeta de la pregunta 2.a. puesto que la observación inicial era que la educación baja tenía, en promedio, un mayor salario.

\section{Pregunta 2.c}

Ver a continuación el análisis de diagnóstico:
\begin{itemize}
	\item Se observó que existen puntos atípicos utilizando la distancia de Cook. Al respecto, se mencionan las observaciones con mayor distancia: 932, 91, 867, 411 y 378.
	\item Se observó, a través del gráfico de cuantiles, que los residuales siguen una distribución normal.
	\item Para cada nivel de educación, se observó los residuales estudentizados. Se observó que la distribución de los residuales son aproximadamente simétricos y con pocos outliers, por lo que podemos deducir que la regresión es homocedástica.
\end{itemize}

\section{Pregunta 2.d}
El estimador puntual del salario para un empleado con un índice de inestabilidad emocional 6.9 y nivel de educación bajo es de 28,214.97.

\end{document}
