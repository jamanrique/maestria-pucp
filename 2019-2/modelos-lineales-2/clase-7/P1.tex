\documentclass{article}
\usepackage[spanish]{babel}
\usepackage[utf8]{inputenc}
\usepackage{amsmath}
\usepackage{amsthm}
\usepackage{amsfonts}
\newtheorem{mydef}{Definition}
\newtheorem{mythm}{Theorem}
\newtheorem{myprf}{Proof}
\title{Modelos lineales}
\author{Justo Andrés Manrique Urbina}
\begin{document}
\maketitle

\section{Pregunta 1}
Sea $Y$ una variable aleatoria discreta con distribución binomial negativa $\mu$ y parámetro de dispersión $\phi$, cuya función de distribución es dada por
\[ f{(y)}=\frac{\Gamma{(y+\phi)}}{\Gamma{(y+1)}\Gamma{(\phi)}}{(\frac{\mu}{\mu+\phi})}^{y}{(\frac{\phi}{\mu+\phi})}^{\phi}, y=0,1,2\ldots\]

Demuestre que para $\phi$ conocido, la distribución de $Y$ pertenece a la familia exponencial.

\textbf{Solución:} La función de probabilidad de $Y$ se puede reexpresar como:
\[ f{(y)}=\exp{(\phi \log{(\frac{\phi}{\mu+\phi})}+y \log{(\frac{\mu}{\mu+\phi})}+\log{(\frac{\Gamma{(y+\phi)}}{y! \Gamma{(\phi)}})})}.\]
\[ f{(y)}=\exp{(\phi \log{(\phi)}+y \log{(n)}-{(\log{(\mu+\phi)})}{(y+\phi)}-\log{(\Gamma{(\phi)})}-\log{(y!)}-\log{(\Gamma{(y+\phi)})})}.\]


\end{document}
