\documentclass{article}
\usepackage[spanish]{babel}
\begin{document}

Tenemos un modelo Y_i = Bo + B1 X + E1. Asumimos que el error tiene distribución normal N(0,sigma cuadrado). El error es independiente. X, por el momento, es conocido. Lo que deberíamos darnos cuenta es que el X es fijo y observable, y el Y_i es aleatorio. Dado ello, Y_i tiene distribución normal(Bo + B1 * Xi, sigma cuadrado). También podemos escribir T_ que tiene una distribución normal (u_i, sigma cuadrado) dónde u_i = N_i y N_i = Bo + B1 X. Existen tres componentes en este modelo.

Distribución de la variable respuesta (parte aleatoria) = Yi ~ N(u_i, sigma cuadrado)
Predictor lineal (parte sistemática) = N_i = Bo + B1 *X_i
Enlace (función de enlace) = u_i = N_i
El modelo lineal es bien flexible. Puede modelar un montón de distribuciones que pueden ser complejas.

OJO: imaginemos que quieres modelar la salario con edad. Este sigue una curva no lineal concava. Podemos pensar el modelo como el siguiente:
Parte aleatoria Yi ~ N(u_i, $\sigma$ cuadrado)
Predictor lineal = N_i = B0, B1 *Xi + B2 * $X_{i}$ al cuadrado
Enlace = u_i = N_i

Podrisa incluso partir la función concava en particiones y modelar la función mediante funciones indicadoras.

Para cada valor de Xi tienes una distribución de la Yi (ver gráfico inkscape).

Imaginemos que queremos modelar conteo. Entonces nuestra variable aleatoria tendría distribución Y ~ pois(u_i), nuestro predictor lineal sería n_i = B0 + B1 X_i) y la función de enlace sería log(u_i) = n_i

Modelo logístico: Y_i ~ Bernoulli(u_i); N_i = Bo + B1_xi; log(u_i/(1-u_i)) = Ni.

Esto da pie a indicar los modelos lineales generalizados. En dónde Y_i ~ FE_familia-exponencial_(u_i,$\phi$), en donde $\phi$ es un parámetro de dispersión.
La familia exponencial es la siguiente: Normal, Gamma, Normal Inversa, Poisson, Bernoulli. Doblemente diferenciable y monótona.

Modelo aditivo generalizado: Asumo que los efectos de las covariables son aditivos.

Todos los modelos que se han mostrado hasta ahora sirven para modelar la media. Por otro lado, la regresión cuantílica sirve para modelar cuantiles.

Se tiene un modelo Yi = B0 + B1 * Xi1 + B2 * Xi2 + ... + Bk Xik + Ei , i= 1, 2, ... , n. Es más práctico evaluar todo de forma matricial.

Se tiene entonces Y = (Y1, Y2, ... , Yn)T, B = (B0, B1, ... , Bk) ,
X =  
(1, X11 .... , X1k
. . . . . . . .  . 
. . . .. . . . . .
1, Xn1, ..... , Xnk)

y E = (e1, e2 ... ek)T

Entonces todo se puede evaluar como Y = XB + E; E ~ N(0, $\sigma$ cuadrado I -identidad-)
El estimador de mínimos cuadrados ordinarios es: $\sum_{i_{n}}$ Ei al cuadardo = (E)T E = (Y-XB)T (Y-XB)
LS(B) = (Y)TY - 2(Y)T XB + (B)T (X)T XB.

Entonces, derivamos la suma de los cuadrados en relación a $\beta$ y sale = -2(X)T Y + 2 (X)T X B = 0
Luego tenemos que ((X)T X)) -1 (X)T X B = ((X)T X) -1 (X)T Y
se cancela el argumento de la izquierda y se tiene que B = ((X)T X) -1 (X)T Y

Luego tenemos que verificar que el estimador es insesgado y con varianza mínima. A este B hay que hallar la esperanza. Utilizamos el TB2 - prueba 2. y obtenemos que: E(B estimado) = E(((X)T X) -1 (X)T Y) = ((X)T X) -1 

ver pantalla de celular. dos imágenes.

El VIF es el factor de inflación de varianza (1/(1-R2 j)).

Vamos a probar que el estimador de B de minimos cuadrados ordinarios es el mejor que hay.

Ver la segunda pantalla de celular. Es un huevo!! XD

El residual no es el error; el residual tiene que ver con el estimado. El error es con el beta de la población.actor de inflación de varianza (1/(1-R2 j)).

Vamos a probar que el estimador de B de minimos cuadrados ordinarios es el mejor que hay.

Ver la segunda pantalla de celular. Es un huevo!! XD
\end{document}
