\documentclass{article}
\usepackage[spanish]{babel}
\usepackage[utf8]{inputenc}
\usepackage{amsmath}
\usepackage{amsthm}
\usepackage{amsfonts}
\newtheorem{mydef}{Definition}
\newtheorem{mythm}{Theorem}
\newtheorem{myprf}{Proof}
\title{Clase 7: Análisis Multivariado}
\author{Justo Andrés Manrique Urbina}
\begin{document}
\maketitle

\section{Análisis Multivariado}
\textbf{Problema:} Dadas dos poblaciones con probabilidad $\pi_{1}$ y $\pi_{2}$ y un elemento $X_{0}={(X_{1}^{0},X_{2}^{0},\ldots,X_{p}^{0})}$. ¿Dónde ubicarlo?

\textbf{Criterio:} Ubicarlo de tal manera que el costo esperado por mala ubicación fuera mínimo.

\subsection{Teoría}
\[ {(\mu_{1}-\mu_{2})}^{T}\Sigma^{-1}x_{0}* \frac{-1}{2} {(\mu_{1}-\mu_{2})}^{T}\Sigma^{-1}{(\mu_{1}+\mu_{2})}\geq 0.\]
\[ {(x_{1}-\bar{x})}^{T}S^{-1}x_{0}> \frac{1}{2}{(\bar{X}_{1}-\bar{X}_{2})}^{T}S^{-1}{(\bar{X_{1}}+\bar{X}_{2}})\]

\section{Escalamiento Multidimensional}

Según el profesor, se usa mucho en marketing. Se tienen muchos individuos $1,2,\ldots,n$ y muchas mediciones de cada individuo $1,2,\ldots,p$
Las redes neuronales convolucionales recorren la imagen en sub-matrices.z

\subsection{Escalamiento Multidimensional No Métrico}
Partimos de una matriz de distancias o similaridades. Esta matriz está compuesta por objetos, es decir las variables descriptivas de cada objeto ya han sido condensadas en esta matriz. Se hace el siguiente procedimiento:
\begin{itemize}
	\item Se obtiene la matriz de similaridades. Solo con la escala ordinal.
	\item Se hace el ordenamiento $S_{i_{1}J_{1}}< \ldots < S_{i_{n}J_{n}}$.
\end{itemize}
\end{document}
