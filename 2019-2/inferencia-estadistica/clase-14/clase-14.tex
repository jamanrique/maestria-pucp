\documentclass{article}
\usepackage[spanish]{babel}
\usepackage[utf8]{inputenc}
\usepackage{amsmath}
\usepackage{amsfonts}

\title{Clase 14: Inferencia Estadística}
\author{Justo Andrés Manrique Urbina}

\begin{document}

\textbf{Definición: } Dadas las siguientes hipótesis:
\[H_{0}:\theta \in \Theta_{0}.\]
\[ H_{1}:\theta \in \Theta_{1}.\]

Fijado $ \eta \text{y} \alpha $. Si $ \forall \theta \in \Theta_{1} $, la regla de decisión óptima para el contraste
\[H_{0}=\theta=\theta_{0}.\]
\[H_{1: \theta=\theta_{2}}.\]
es la misma, entonces esta regla se llama la regla de decisión uniformemente más poderosa.

\section{Ejemplo}
\[ X\sim N(\mu,10).\]
en dónde $ n=20 $ y $ \alpha=0.05 $. Deduzca, si existe, la regla UMP para el contraste:
\[H_{0}: \mu=5.\]
\[H_{1}: \mu>5.\]

\textbf{Solución:} Tenemos que:
\[\Theta_{1=\{\mu:\mu>5\}}.\]
\[H_{0}:\mu=5.\]
\[H_{1}: \mu \in \Theta_{1}.\]

Sea $\mu_{1} > 5 (\mu \in \Theta_{1})$ y consideramos el contraste:
\[H_{0}: \mu=5.\]
\[H_{1}:\mu=\mu_{1}.\]

Mediante Neyman-Pearson hallamos la regla de decisión óptima para estas hipótesis (es decir, la que minimiza $\beta$ o maximiza la potencia $1-\beta$). Se vió en un ejemplo anterior que:

\[l(\mu)=20(-\ln(\sqrt{2\pi}-\ln(\sqrt{10})))-\frac{1}{10} \sum^{20}_{j=1}{(X_{j}-\mu)}^{2}.\]

Se rechaza $H_{0}$ si:
\[\frac{L(\mu_{1})}{L(5)}>5.\]
\[20(-\ln(\sqrt{2\pi}-\ln(\sqrt{10})))-\frac{1}{10}\sum^{20}_{1}{(X_{j}-\mu_{1})}^{2}-20(-\ln(\sqrt{2\pi})-\ln(\sqrt{10}))+\frac{1}{10}\sum{(X_{j}-5)}^{2} > c.\]

\[(\mu_{1}-5)\sum^{20}_{1}{(X_{j}-5)}^{2}>c.\]
\[\sum^{20}_{1}X_{j} > c.\]
Pues no depende de $\mu_{1}$
\[\bar{X}>c.\]

Dónde $c$ es tal que:
\[\text{P (Rechazar $H_{0}$ siendo verdadera)}=0.05.\]
\[P(\bar{X}> c, \mu=5)=0.05.\]

Por otra parte, tenemos que:
\[X\sim N(\mu,\sigma^{2}).\]
\[Z \sim N(0,1).\]

\[Z\sim \frac{\sqrt{20}(\bar{X}-\mu)}{\sqrt{10}}\sim N(0,1).\]

En dónde:
\[F_{z}(\frac{\sqrt{20}(c-5)}{\sqrt{10}})=0.95.\]
\[\frac{\sqrt{20}(c-5)}{\sqrt{10}}=1.645.\]
\[c=6.1632.\]

Se rechaza $H_{0}$ si $\bar{X}>6.1632$. Esta es la regla de decisión óptima (minimiza $\beta$ o maximiza $\pi$). Como esta regla de decisión es la misma para cualquier valor de $\mu_{1}$ (no depende de $\mu_{2}$) entonces esta es la regla de decisión uniformemente más poderosa.

\section{Ejercicio}

En el ejemplo anterior determinar $\beta$ (para el contraste original). \textbf{Solución:}
\[\beta=P(\text{Aceptar $H_{0}$ siendo falsa}).\]
\[\beta=P(\bar{X}\leq 6.1632, \mu > 5).\]
\[\beta=\beta(\mu)=P(\bar{X} \leq 6.1632), \mu>5.\]
\[=F_{\bar{X}}(6.1632),\mu>5.\]

En relación al ejemplo anterior, tenemos que:
\[\beta=\beta(\mu)=F_{z}(\frac{\sqrt{20}(6.1632-\mu)}{\sqrt{10}}),\mu>5.\]

Note lo siguiente:

\begin{itemize}
	\item Si $\mu$ sube, entonces $\beta(\mu)$ baja.
	\item $\beta(\infty)=\lim_{\mu \rightarrow \infty}\beta(\mu)=\lim_{\mu \rightarrow \infty} F_{z}(\frac{\sqrt{20}}{\sqrt{10}})(6.163-\mu)=0$
	\item $\beta(5^{+}) = \lim_{\mu \rightarrow 5^{+}} \beta(\mu) = F_{z}(\frac{\sqrt{20}(6.1632-5)}{\sqrt{10}})=F_{z}(1.645)=0.95$.
\end{itemize}

Sea $\mu_{1}<5$ y considerando el contraste auxiliar:
\[H_{0}: \mu=5.\]
\[H_{1}: \mu=\mu_{1}.\]
A este contraste se le aplica el Lema de Neyman-Pearson.

Se rechaza $H_{0}$ si:
\[\frac{L(\mu_{1})}{L(5)}>c.\]
\[l(\mu_{1})-l(5)>c.\]
\[\sum^{20}_{1}X_{j} < c.\]

Se rechaza $H_{0}$ si $\bar{X} < c$ dónde $c$ es tal que:
\[P(\text{Rechazar $H_{0}$ siendo verdadera})=0.05.\]
\[P(\bar{x}<c, \mu=5)=0.05.\]
\[F_{\bar{X}}(c)=0.05, \mu=5.\]

Pero como:

\[Z=\frac{\sqrt{20}(\bar{X}-\mu)}{\sqrt{10}}\sim N(0,1).\]

Resulta que:
\[F_{z}(\frac{\sqrt{20}(c-5)}{\sqrt{10}})=0.05.\]
\[\frac{\sqrt{20}(c-5)}{\sqrt{10}}=1.645.\]
\[c=3.8368.\]

Se rechaza $H_{0}$ si $\bar{X}< 3.8368$

Esta regla es la regla óptima (minimiza $\beta$) para el contraste auxiliar, que no depende de $\mu_{1}$, viene a ser la regla de decisión uniformemente más poderosa para el contraste de hipótesis regional.

\section{Ejercicio}
En el ejemplo anterior graficar $\beta$. Sea $\beta$ tal que:
\[\beta=P(\text{Aceptar $H_{0}$ siendo falsa.}).\]
\[=P(\bar{X}>3.8368, \mu<5).\]
\[\beta=\beta(\mu)=P(\bar{X}>3.8368),\mu=5.\]
\[=1-F_{\bar{X}}(3.8368),\mu=5.\]
\[H_{0}: \mu=5.\]
\[H_{1}=\mu<5.\]

Estandarizando:
\[\beta=\beta(\mu)=1-F_{z}(\frac{\sqrt{20}}{\sqrt{10}}(3.8368-\mu)).\]

Observar que:
\begin{itemize}
	\item Si $\mu$ sube, entonces $\beta(\mu)$ sube.
	\item $\beta(-\infty)=\lim_{\mu \rightarrow \infty} \beta(\mu)=1-F_{z}(\infty)=0$
	\item $\beta(5^{-})=\lim_{\mu \rightarrow 5^{-}}\beta(\mu)=1-0.05=0.95$
\end{itemize}

\section{Ejemplo 3}
\[X\sim N(\mu,10), n=20, \alpha=0.05.\]

\[H_{0}:\mu=5.\]
\[H_{1}:\mu \neq 5.\]
Sea $\mu \neq 5$ y consideramos el contraste de hipótesis auxiliar.

\end{document}
