\documentclass{article}
\usepackage[spanish]{babel}
\usepackage[utf8]{inputenc}
\usepackage{amsmath}
\usepackage{amsthm}
\usepackage{amsfonts}
\newtheorem{mydef}{Definition}
\newtheorem{mythm}{Theorem}
\newtheorem{myprf}{Proof}
\title{Clase 5: Inferencia Estadística}
\author{Justo Andrés Manrique Urbina}
\begin{document}
\maketitle

\section{Ejemplo}
\[ X \sim U(0;\theta).\]
Definamos $f_{\tilde{X}}=\frac{1}{\theta^{n}}, X_{{(n)}} \leq \theta; 0, X_{{(n)}} > 0$
\[ \frac{f_{\tilde{X}}{(X)}}{f_{\tilde{X}}}{(Y)}=1; Y_{{(n)}}\leq \theta, X_{{(n)}}\leq \theta \text{ o } 0; Y_{{(n)}} \leq \theta, X_{{(n)}}>\theta.\]
\[ X_{{(n)}}=Y_{{(n)}} \rightarrow \frac{f_{{\tilde{X}}}{(X)}}{f_{{\tilde{X}}}{(Y)}}=1, X_{{(n)}} \leq \theta.\]

Sea $X_{{(n)}}<Y_{{(n)}}$:
\[ \rightarrow Y_{{(n)}} \leq \theta \rightarrow X_{{(n)}} \leq \theta \rightarrow \frac{f_{\tilde{X}}{(X)}}{f_{\tilde{X}}{(Y)}=1, Y_{{(n)}}} \leq \theta.\]

Por lo tanto, no se puede afirmar que $X_{{(n)}}$ sea minimal.

\section{Ejemplo}
\[ X\sim P(\lambda), \lambda > 0.\]

\[ f_{\tilde{X}}(\tilde{X})=\frac{1}{x_{1}!x_{2}!\ldots x_{n}!}\lambda^{\sum_{i=1}^{n}X_{i}}e^{-n\lambda}.\]

\[ \frac{f_{\tilde{X}}{(\tilde{X})}}{f_{\tilde{X}}{(\tilde{Y})}}=\frac{y_{1}!y_{2}!\ldots y_{n}!}{x_{1}!x_{2}!\ldots x_{n}!}\lambda^{\sum x_{i} - \sum y_{i}}.\]

\begin{mydef}
	\textbf{Completitud: } La familia de distribuciones $\{f{(x;\theta), \theta \in \Theta}\}$ es completa si
	\[ X\sim f{(x;\theta)}, \theta \in \Theta.\]
	\[ E{(U{(X)})}=0 \rightarrow U{(X)} = 0\text{, c.s.}.\]
	\[  P{(U{(X)}=0)}=1.\]
\end{mydef}

\section{Ejemplo}
La familia de distribuciones binomial es completa:
\[ f{(x,\theta)}= \binom{N}{x} \theta^{x}{(1-\theta)}^{n-x},x=0,1,\ldots,n.\]
Si $X \sim f{(x,\theta)}$:
\[ E{(U{(X)})}=\sum_{x=0}^{n}U{(X)}\binom{N}{x}\theta^{x}{(1-\theta)}^{n-x}.\]
Luego, $E{(U{(X)})}=0, \forall \theta \in {(0;1)}$

\[ U{(0)}{(1-\theta)}^{n-1}+U{(1)}n\theta{(1-\theta)}^{n-1}+\ldots+I{(n)}\theta^{n}.\]

Este es un polinomio en $\theta$ de grado $\leq n$. Sin embargo existirían infinitas raíces si $\theta \in {(0,1)}$, por lo que necesariamente $U{(X)}=0$.

\begin{mydef}
\textbf{Completitud: } Una estadística T es completa si su familia de distribuciones es completa.
\end{mydef}

\begin{mythm}
En la familia exponencial, si $\Theta$ es un conjunto abierto, entonces las estadísticas suficientes son completas.
\end{mythm}

\section{Ejemplo}
\[ X\sim \exp{(\theta)},\theta > 0; \Theta \in \mathbb{R}.\]
\[ T=\sum_{i=1}^{n}X_{j} \text{ es suficiente y} \Theta=\mathbb{R}^{+}\text { es un conjunto abierto}.\]
Por lo tanto T es completa.
\end{document}
