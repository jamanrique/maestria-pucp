\documentclass{article}
\usepackage[spanish]{babel}
\usepackage[utf8]{inputenc}
\title{Clase 1: Inferencia Estadística}
\maketitle
\begin{document}
Primera definición: Sean $X_{1},\ldots,X_{n}$ variables aleatorias (definidas en el espacio muestral) a partir de las cuales se realizará la inferencia. Si estas variables son independientes y tienen la misma distribución de $X$, se dice que se tiene una muestra aleatoria simple de $X$.
Segunda definición: El parámetro es una cantidad $\theta$, que depende de la distirbución de la muestra disponible $f_{X_{1},\ldots,X_{n}}$. $\theta$ es desconocido, pues sus valores posibles constituyen el conjunto $\Theta$. Cabe resaltar que $\theta$ también puede ser un vector.
\section{Ejemplo 1}
Sea $Y_{1}=\theta X_{1} + \epsilon_{1}, \ldots, Y_{n}=\theta X_{n} + \epsilon_{n}$ dónde $X_{1},\ldots,X_{n}$ son cantidades conocidas. Asimismo $\theta \in \mathbb{R}$ es una cantidad desconocia y $\epsilon_{1},\ldots,\epsilon{n}$ son variables aleatorias independientes y cada una tiene una distribución $N(0,1)$. $Y_{1}, \ldots, Y_{n}$ es una muestra aleatoria. Estas variables son independientes; sin embargo no tienen la misma distribución, $Y_{i}~N(\theta X_{i},1)$.
\section{Ejemplo 2: Inferencia sobre una población}
Sea $X_{i}=1$, si la unidad de observación i satisface cierta característica; 0, caso contrario para $i=1,\ldots,n$. Sea $P(X_{i}=1)=\theta, i=1,\ldots,n$ y $P(X_{i}=0)=1-\theta, i=1,\ldots,n$ Entonces $X_{i}~B(\theta)$.
Si asumimos que $X_{1},\ldots,X_{n}$ son independientes, entonces $X_{1},\ldots,X_{n}$ es una muestra aleatoria simple de $X$ en dónde $X~B(\theta)$. Interpretación: $\theta$= proporción de unidades, en una población, que satisface la característica.
Tercera definición: La estadística es cualquier función de la muestra (disponible) $g(X_{1},\ldots,X_{n})$. Usualmente $g$ es una función de valor real, pero también puede ser de valor vectorial. \textbf{Observación:} $g$ no debe depender de parámetros desconocidos.
\section{Ejemplos de la tercera definición}
$X_{(1)}=minimo{X_{1}, \ldots, X_{n}}$
$X_{(1)}=maximo{X_{1}, \ldots, X_{n}} $
$\sum_{j=1}=X_{j}$
$\frac{\sum_{j=1}=X_{j}}{n}$ media muestral, también existe la varianza muestral y el segundo momento muestral.
Cuarta definición: Si $\theta$ es un parámetro, un estimador de $\theta$ es una estadística $\hat{\theta}= \hat{\theta}(X_{1},\ldots,X_{n})$ que se usa para aproximar el valor desconocido de $\theta$. \textbf{Observación:} $\hat{\theta}$ es una variable aleatoria.
\section{Propiedades de un estimador}
\subsection{Propiedad de insesgamiento}
$\hat{\theta}$ es un estimador insesgado de $\theta$, si $E(\hat{\theta})=\theta, \forall \theta \in \Theta$.
\subsection{Ejemplo de insesgamiento}
El Estimador usual de $\theta$ está dado por:
\[  \hat{\theta} = \frac{\sum_{j=1}X_{j}Y_{j}}{\sum_{j=1}X_{j}X_{j}}.\]
\[  E(\hat{\theta})=  \frac{\sum_{j=1}X_{j}E(Y_{j})}{\sum_{j=1}X_{j}X_{j}} = \theta.\]
Dado que $E(Y_{j}) = \theta X_{j}$. Por lo tanto, $\hat{\theta}$ es un estimador insesgado de $\theta$.

En general se tiene que
\[ E(\bar{g(x)}) = E(g(x)).\]
\textbf{Demostración:}
\[ \bar{g(x)}= \frac{\sum_{j=1}g(X_{j})}{n}.\]
\[ E(\bar{g(x)})=  \frac{1}{n} \sum_{j=1} E(g(X_{j})).\]
\[ E(\bar{g(x)})=  \frac{1}{n} n \sum_{j=1} E(g(X)).\]
\[ E(g(X)).\]
El estimador usual de la varianza $\sigma^{2}$ es:
Sea una muestra aleatoria simple de $X$ dónde $E(X)=\mu$ y $V(X)=\sigma^{2}$. $S^{2}$ es el estimador usual de $\sigma^{2}$. 

Ver imagen

\section{Ejemplo}
Sea $E(\bar{X^{2}}) = V(\bar{X}) + E^{2}(\bar{X})$. Entonces se tiene que:
\[ V(\frac{\sum_{j=1}^{n}X_{j}}{n}) + \mu^{2}.\]
\[ \frac{1}{n^{2}} \sum_{j=1}^{n}V(X_{j}) + \mu^{2}.\] Por independencia se tiene que $V(\sum) = \sum V()$.
\[  \frac{\sigma^{2}}{n} + \mu^{2}.\]
Luego, de (1), (2) y (3):
\[ E(S^{2})=  \frac{n}{n-1}[E(X^{2})- \ldots ].\]
\[  \frac{n}{n-1}[\sigma^{2}+\mu^{2}-( \frac{\sigma^{2}}{n} + \mu^{2})].\]
\[ \sigma^{2}.\]

Definición: Si $\lim_{n infinito} E(\hat{\theta})=\theta, \forall \theta \in \Theta$, se dice que $\hat{\theta}$ es asintóticamente insesgado.
\section{Propiedad de eficiencia}
Si $\hat{\theta_{1}}$ y $\hat{\theta_{1}}$ son estimadores insesgados de $\theta$, se dice que $\hat{\theta_{1}}$ es más eficiente que  $\hat{\theta_{1}}$ si $V(\hat{\theta_{1}}) < V(\hat{\theta_{2}})$.
\textbf{Explicación:}
\end{document}
