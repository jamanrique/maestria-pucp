\documentclass{article}
\usepackage[spanish]{babel}
\usepackage[utf8]{inputenc}
\usepackage{amsmath}
\usepackage{amsthm}
\usepackage{amsfonts}
\newtheorem{mydef}{Definition}
\newtheorem{mythm}{Theorem}
\newtheorem{myprf}{Proof}
\title{Clase 4: Inferencia Estadística}
\author{Justo Andrés Manrique Urbina}
\begin{document}
\maketitle
\section{Parte 1}
Sea $X=(X_{1},X_{2},\ldots,X_{n})$ y que:
\[ \forall t: \{ x: T_{2}{(x)}=t\} \in \{x: T_{1}{(x)}=t_{1}\}, \exists t_{1}.\]
Una definición equivalente es:
\[ T_{2}{(x)}=T_{2}{(y)} \rightarrow T_{1}{(x)}=T_{1}{(y)}.\]
Otra definición equivalente es:
\[ T_{1}=h{(T_{2})}.\]

\subsection{1 implica 2}
\[ T_{2}{(x)}=T_{2}{(y)}.\]
Sea $t=T_{2}{(x)}=T_{2}{(y)}$
\[ \rightarrow x \in \{z:T_{2}(z)=t\}\in \{z:T_{1}{(z)}=t_{1}\},\exists t_{1}.\]
\[ \rightarrow y \in \{z:T_{2}{(z)}=t\} \in \{z: T_{1}{(z)}=t_{1}\},\exists t_{1}\]
\subsection{3 implica 2}
\[ T_{2}{(x)}=T_{2}{(y)}.\]
\[ \rightarrow h(T_{2}{(x)})=h{(T_{2}{(y)})}.\]
\[ \rightarrow T_{1}{(x)}=T_{2}{(y)}.\]

\subsection{2 implica 3}
\[ t \in \mathbb{R}_{T_{2}}.\]
\[ \rightarrow \exists x_{0} tq T_{2}{(x_{0})}=t.\]
\[ h(t)=T_{1}{(x_{0})}.\]
Notemos que si existe $y_{9}\neq x_{0}$ tal que $T_{2}{(x_{0})}=t$, entonces se tiene que:
\[ T_{1}{(X_{0})}=T_{2}{(y_{0})}.\]
\[ h{(t)}=T_{1}{(X_{0})}=T_{2}{(y_{0})}.\]

\section{Estadística suficiente y mínima}
\begin{mydef}
$T$ es una estadística suficiente y mínima  si resume tanto o más que cualquier otra estadística suficiente.
\end{mydef}

\textbf{Observación: }Si $T$ es suficiente y mínima y $T_{2}$ es suficiente, entonces se cumple que:
\begin{itemize}
	\item \[ \forall t:\{x:T_{2}{(x)}=T\} \in \{x:T(X)=T_{x}\}, \exists t_{1}.\]
	\item \[ T_{2}{(x)}=T_{2}{(y)}\rightarrow T{(x)}=T{(y)}.\]
	\item  \[\exists h: T{(x)}=h{({(T_{{(x)}})})}\]
\end{itemize}

\begin{mythm}
$T$ es una estadística suficiente y mínima si y solo si se cumple que:
\[ \frac{f_{(X_{1},X_{2},\ldots,X_{n})}{(z)}}{f_{(X_{1},X_{2},\ldots,X_{n})}{(y)}}.\]
esto es independiente de $\theta$.
\end{mythm}
\section{Ejemplo}
Sea
\[ X\sim \text{doble exponencial($\theta$)}.\]
y su función de densidad está dada por:
\[ f_{{(x)}}=\frac{1}{2}\theta e^{-\theta |x|}, \forall x \in \mathbb{R}.\]
La función conjunta de $f_{c}{(x)}$ $(X_{1},X_{2},\ldots,X_{n})$ está dada por:
\[ {(\frac{1}{2})}^{n}\theta^{n}e^{-\theta \sum_{j=1}^{n} |X_{j}|}.\]
Para hallar si la estadística es suficiente y mínima entonces utilizamos el cociente:
\[ \frac{f_{c}{(x)}}{f_{c}{(y)}}=e^{-\theta{(\sum |x_{j}| - \sum |y_{j}|)}}.\]
Si ambos son iguales, entonces es independiente de $\theta$.

\section{Ejemplo}
Sea
\[ X \sim N(0;\sigma^{2});\theta=\sigma^{2}>0.\]
Entonces la función conjunta es:
\[ {(\frac{1}{\sqrt{2\pi}})}^{n}\sigma^{-n}e^{-\frac{1}{2\sigma^{2}}\sum X_{j}^{2}}.\]
Si se utiliza el teorema, entonces se tiene:
\[ \frac{f_{c}{(x)}}{f_{c}{(y)}}=e^{-\frac{1}{2\sigma^{2}}{(\sum X_{j}^{2}-\sum Y_{j}^{2})}}.\]
Si ambos son iguales, entonces es independiente de $\sigma^{2}$.
\end{document}
