\documentclass{article}
\usepackage[spanish]{babel}
\usepackage[utf8]{inputenc}
\usepackage{amsmath}
\usepackage{amsthm}
\usepackage{amsfonts}
\newtheorem{mydef}{Definition}
\newtheorem{mythm}{Theorem}
\newtheorem{myprf}{Proof}
\title{Clase 7: Inferencia Estadística}
\author{Justo Andrés Manrique Urbina}
\begin{document}
\maketitle

\section{Desigualdad de Crámer-Rao}

Si se cumplen las condiciones de regularidad, para todo $\hat{\theta}$ estimador insesgado de $\theta$:
\[ V{(\hat{\theta})} \geq \frac{1}{n I_{F}{(\theta)}}.\]

\subsection{Ejemplo}

Sea $X \sim B{(\theta)}$:
\[ I_{F}{(\theta)}=\frac{1}{\theta{(1-\theta)}}.\]
Si $\hat{\theta}$ es un estimador insesgado. Entonces, por desigualdad de Crámer-Rao:
\[ V{(\hat{\theta})} \geq \frac{\theta{(1-\theta)}}{n}.\]

\begin{itemize}
	\item $\bar{X}$ es un estimador insesgado. $E{(\hat{X})}=E{(X)}=\theta, \forall \theta \in {(0;1)}$
	\item $V{(\hat{X})}=\frac{V{(X)}}{n}=\frac{\theta{(1-\theta)}}{n}$
\end{itemize}

\end{document}
